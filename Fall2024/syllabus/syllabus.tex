\documentclass[11pt,letterpaper]{article}
\usepackage{array}
\usepackage{fullpage}
\usepackage{verbatim}
\usepackage{parskip}
\usepackage{array}
\usepackage{calc}
\usepackage{url}
\usepackage{hyperref}
\newcommand{\squeezeup}{\vspace{-2.5mm}}
\setlength{\parindent}{0in}
\newcommand{\tablespace}[0]{\vspace{8pt}}

\usepackage{titlesec}
\titleformat{\section}{\normalfont\bfseries}{\thesection}{0em}{}
\titlespacing{\section}{0pt}{3pt}{0pt}

\titleformat{\subsection}{\normalfont\bfseries}{\thesubsection}{0em}{}
\titlespacing{\section}{0pt}{11pt}{0pt}

\titleformat{\subsubsection}{\normalfont\bfseries}{\thesubsubsection}{0em}{}
\titlespacing{\subsubsection}{0pt}{3pt}{0pt}


\begin{document}
\begin{centering}
\textbf{ENVS S422: Earth's Climate System}

Fall 2024

\bigskip
\begin{table}[h]
\centering
\setlength{\extrarowheight}{2pt}
\squeezeup
\begin{tabular}{@{}r@{\hspace{0.1in}}p{4.25in}} 
{\bf Instructor:} & Jason Amundson\\
& {\'A}ak'w T{\'a} H{\'i}t 209\\
& jmamundson@alaska.edu\\
& phone: 796-6247 \tablespace\\
{\bf Class hours:} & MW 1:45 pm -- 3:15 pm
\tablespace\\
{\bf Office hours:} & MWF 12:00 pm -- 1:00 pm or by appointment\tablespace\\
{\bf Website:} & A course website will be maintained on Blackboard (\url{http://classes.alaska.edu}). Check for assignments, grades, and messages.\tablespace\\
{\bf Prerequisites:} & ENVS S102 and PHYS S103 or S211, or instructor permission\tablespace\\
{\bf Textbook:} & The Earth System, 3\textsuperscript{rd} ed. \\
& by Kump, Kasting, and Crane (ISBN: 978-0-321-59779-3)\\
 & \underline{Not} required, and on reserve at the library circulation desk \tablespace\\
{\bf Other materials:} & Additional supplementary reading material will be provided throughout the semester.\tablespace
\end{tabular}
\end{table}
\end{centering}

\section*{Course overview}
This course explores the processes controlling Earth's climate and looks at how these processes have contributed to past and contemporary climate change. We will start with simple climate models and gradually add layers of complexity. Emphasis will be placed on understanding the feedbacks between the various components of Earth's climate system.

\section*{Student Learning Outcomes}
In this course students will learn to
\begin{enumerate}\itemsep -5pt
\item describe key components, interactions, and concepts of the Earth system
\item analyze the causes of climate change over various temporal and spatial scales
\item build simple models of Earth system interactions
\item critically read and discuss current literature in Earth system science
\end{enumerate}


\clearpage
\textbf{Grading} 
\begin{table}[h!]
\squeezeup
\begin{tabular}{ll}
Class participation & 15\%\\
\hspace{15pt}\textit{presentation} &\hspace{15pt}\textit{5\%}\\
\hspace{15pt}\textit{discussion} & \hspace{15pt}\textit{10\%}\\
Modeling exercises & 15\%\\
Term paper & 40\%\\
\hspace{15pt}\textit{proposal} & \hspace{15pt}\textit{5\%}\\
\hspace{15pt}\textit{final draft} & \hspace{15pt}\textit{25\%}\\
\hspace{15pt}\textit{presentation} & \hspace{15pt}\textit{10\%}\\
Midterm exam & 15\%\\
Final exam & 15\%
\end{tabular}
\end{table}

Grades for this class will be based on class participation, modeling exercises, a term paper, and two exams.

Class participation: We will read and discuss approximately one paper per week. For each paper you will be asked to submit a short summary of the paper that will count toward your participation grade. In addition, you will be expected to present and lead a discussion on one of the articles (listed in bold in the class schedule). The discussions will be held in a ``round table'' format and should last 20--30 minutes. 

Modeling exercises: We will use the STELLA software package to model various components of the Earth system. You will be required to submit brief reports on these exercises. STELLA is available on the university computers, which you can access from your computer by using VMware and navigating to \url{https://mydesktop.uas.alaska.edu/}.

Term paper: You will write a 15-20 page term paper on a topic of your choosing. The topic must span at least two components of the Earth system. You are asked to submit a proposal (1 page and a list of references) and a final draft. I am happy to read through drafts of the paper to give you feedback. You will also present the paper during a 10--15 minute talk during finals week.

Exams: The mid-term and final exam will be essay exams done outside of class.

\textbf{Late policy:} With the exception of exams, homework will not be considered late up until the point at which I start grading the assignment. Anything submitted after that point will be given a maximum grade of 50\% and will be given less feedback. The exams will not be accepted after their due dates.

\clearpage
\section*{Grading Scale}
\begin{table}[h!]
\squeezeup
\begin{tabular}{ll}
A & 93--100\% \\
A- & 90--92\% \\
B+ & 87--89\% \\
B & 83--86\% \\
B- & 80--82\% \\
C+ & 77--79\% \\
C & 73--76\% \\
C- & 70--72\% \\
D+ & 67--69\% \\
D & 63--66\% \\
D- & 60--62\% \\
F & $<$60\%
\end{tabular}
\end{table}

I may lower this grading scale if I decide that the course assignments have been too difficult. I will not do the opposite.


\section*{Student Ratings of Instruction}
During the last three weeks of class, you will have an opportunity to complete an on-line rating questionnaire on course instruction, how the course aided in your skill development,  effectiveness of technology and equipment used, and adequacy of library resources and services used during the course. You will receive notification in your UAS email account when the rating questionnaire is available. Please make use of this opportunity to provide feedback on what worked for you and what did not. Your input is used to assess methods and services in order to provide the best educational experience possible.

\section*{Disabilities}
If you experience a disability and would like information about support services, please contact Disability Services, located at the Student Resource Center in the Mourant building.  They can be reached at 796-6000. For more information, please see \url{http://www.uas.alaska.edu/dss/index.html}.

\section*{Title IX/Sexual Misconduct}
All students have the right to be free from all forms of gender and sex-based misconduct (sexual harassment, dating violence, domestic violence, sexual assault, or stalking). Please report any incidence of sex or gender-based discrimination to the UAS Title IX Office: {https://uas.alaska.edu/equity-and-compliance/titleix/index.html}.


\clearpage
%\section*{Tentative schedule}
\begin{table}[ht]
{\bf Tentative schedule}\tablespace\\% (subject to change)\tablespace\\
\setlength{\extrarowheight}{2pt}
\begin{tabular}{>{\centering}p{0.1\textwidth-\tabcolsep} p{0.55\textwidth-\tabcolsep} >{\raggedright\arraybackslash}p{0.3\textwidth-\tabcolsep}}
%\begin{tabular}{@{}r@{\hspace{0.1in}}p{4.25in}} 
Date & Topic(s) & Reading material\\
\hline
8/26 & Current and past climate change & ch. 1; IPCC SPM \\
8/28 & MODELING: Introduction to STELLA & \\
9/4 & Introduction to systems and Daisyworld & ch. 2\\
9/9 & MODELING: Daisyworld & \\
9/11 & Global energy balance and the greenhouse effect & ch. 3; {Kirk-Davidoff, 2018}\\
9/16 & MODELING: The global energy budget & \\
9/18  & Atmosphere: Meridional circulation, the Coriolis effect, and seasonal variability & ch. 4 \\
9/23  & Atmosphere: Effect of land masses &  ch. 4; {\bf Molnar et al., 2010 (skip section 2)}\\
9/25 & Oceans: Winds and surface currents; El Ni\~no & ch. 5 \\ % include El Nino
9/30 & Oceans: Thermohaline circulation & ch. 5; {\bf Rahmstorf, 2002}\\
10/2 & MODELING: Thermohaline circulation & \\
10/7 & Cryosphere: Ice-albedo feedback & ch. 6; {\bf Eisenman and Wettlaufer, 2009}\\
10/9 & MODELING: Ice sheets &\\
10/14 & Cryosphere: Ice-ocean interactions & {\bf Bassis et al., 2017} \\
10/16  & Solid Earth: Mantle convection and plate tectonics & ch. 7\\
10/21  & Solid Earth: Erosion and mountain building & {\bf Pedersen and Egholm, 2013} \\
10/23  & MODELING: Rock cycle & \\
10/28 & Carbon cycle & ch. 8; {\bf Schuur et al., 2015}\\
10/30 & MODELING: Carbon cycle & \\
11/4 & Hydrologic cycle & ch. 4; {\bf Taylor et al., 2013}\\
11/6 & MODELING: Hydrologic cycle & \\
11/11 &  Role of the biosphere & ch. 9; {Inside Science article} \\
11/13 & Early Earth--Snowball Earth & chs. 10--12; {Donnadieu et al., 2004} \\
11/18 & Paleozoic and Mesozoic \\
11/25 & Cenozoic climate and Pleistocene glaciations & ch. 14; {Huybers and Wunsch, 2005} \\
12/2 & The Holocene; course review & ch. 15; {Condron and Winsor, 2012}\\
12/4 & Presentation of term papers (1:00 pm -- 3:00 pm) & \\ 
\hline
\end{tabular}
\end{table}


\clearpage
\begin{table}[ht]
{\bf Important Due Dates}\tablespace\\
\setlength{\extrarowheight}{2pt}
\begin{tabular}{cl}
\hline
10/14 & term paper proposal\\ 
10/18 & midterm exam\\
12/6 & final draft of term paper\\
12/13 & final exam\\
\hline
\end{tabular}
\end{table}

\end{document}
