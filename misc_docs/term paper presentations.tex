\documentclass[11pt,letterpaper]{article}
\usepackage{array}
%\usepackage[top=1in, bottom=1in, left=1in, right=1in]{geometry}
\usepackage{fullpage}
\usepackage{verbatim}
\usepackage{parskip}
\usepackage{graphicx}
\usepackage{makecell}
\newcolumntype{?}{!{\vrule width 1pt}}
\usepackage{tabularx}
%\pagestyle{empty}

\newcommand{\squeezeup}{\vspace{-2.5mm}}

\begin{document}
\setlength{\parindent}{0in}
%\baselineskip 4pt
\newcommand{\tablespace}[0]{\vspace{8pt}}


\textbf{ENVS S422}\\
Term paper presentations\\

\hrule
You will be presenting your term papers in a 12--15 minute talk during the final class period. Treat the presentation as if you were presenting a research project at a conference. 12--15 minutes is not a lot of time, so you will need to think carefully about how to organize and present your topic. You should be well prepared so that you can talk comfortably and with confidence.

Your goal in a presentation like this is to capture the audience's attention and explain to them why your topic is interesting and relevant. In other words, don't simply read notes word-for-word. This is boring and indicates that you aren't familiar with the subject. I find that the best presentations are the ones in which the presenter (1) emphasizes a few main points and (2) tells a ``scientific story''. Too much detail and the audience may lose focus; not enough detail and the the story is hard to follow. Don't worry about justifying all of your arguments unless there are some important arguments that need to be made (and haven't been made previously). Obviously, there is some gray area here. Giving good presentations takes a lot of practice.

Many people start their presentations with an outline slide. I'm not a huge fan of outline slides---for short scientific talks, anyway---because they take up too much time and often only say that the topic is going to be introduced and some conclusions are going to be drawn. That said, it is a good idea to give the audience a general idea of where you are going with the talk. (I sometimes like to give away the punchline at the beginning of the talk by saying something like ``In this talk I will demonstrate...'') Make sure that you have some important conclusions at the end, but just focus on two or three of the most important parts of your project. Keep in mind that your audience likely doesn't understand your subject as well as you do.

\clearpage
Grading rubric for presentation of term papers

Name:
\vspace{1cm}

\begin{table}[h]
\begin{tabularx}{\textwidth}{p{1in}?X|X|X|}
  & \textbf{Exceeds Standards} & \textbf{Meets Standards} & \textbf{Standards Not Met} \\ \Xhline{2\arrayrulewidth}
  \textbf{Presentation length} & Within time limits (12 min $< t <$ 15 min)\hspace{1cm} (4 pts) & Within a few minutes of the time limits ($t<$8 min or $t>$12 min) \hspace{1cm}(2 pts) & Very short ($t<$9 min) or very long ($t>$18 min) \hspace{1cm}(0 pts)\\ \hline
  \textbf{Motivation} & Clearly motivated by Earth system processes (2 pts) & Motivation is implied or not tightly connected to Earth system processes (1 pt) & No motivation / connection to course topics \hspace{1cm}(0 pts) \\ \hline
  \textbf{Organization} & Very clear, logical structure with strong arguments (4 pts) & Some parts of the presentation were disorganized; arguments don't build toward conclusions (2 pts) & Very disorganized and no clear arguments\hspace{1cm}(0 pts) \\ \hline
  \textbf{Factually correct} & All aspects of the presentation were factually correct (4 pts) & Some incorrect statements (2 pts) & Numerous incorrect statements (0 pts) \\ \hline
  \textbf{Summary} & 2--3 key points are clearly articulated \hspace{1cm}(2 pts) & Too few or too many key points or not clearly articulated (1 pt) & No summary or key points presented (0 pts)  \\ \hline
  \textbf{Design} & Presentation visually appealing with minimal text; easy to understand (2 pts) & Some slides are difficult to follow (1 pt)  & Most/all slides are difficult to understand \hspace{1cm}(0 pts) \\ \hline
  \textbf{Voice} & Spoken in a clear voice with confidence; did not read slides/notes verbatim (2 pts) & Struggled with some aspects of the presentation (1 pt) & Not well-prepared or confident; read notes directly (0 pts) \\ \hline
\end{tabularx}

\vspace{1cm}
Total: \hspace{1.5cm} /20
\vspace{1cm}

Comments:

\end{table}
\end{document}