\documentclass[11pt,letterpaper]{article}
\usepackage{array}
%\usepackage[top=1in, bottom=1in, left=1in, right=1in]{geometry}
\usepackage{fullpage}
\usepackage{verbatim}
\usepackage{parskip}
\usepackage{graphicx}
\usepackage{makecell}
\newcolumntype{?}{!{\vrule width 1pt}}
\usepackage{tabularx}
%\pagestyle{empty}

\newcommand{\squeezeup}{\vspace{-2.5mm}}

\begin{document}
\setlength{\parindent}{0in}
%\baselineskip 4pt
\newcommand{\tablespace}[0]{\vspace{8pt}}
\textbf{ENVS S422}\\
Term paper guidelines\\

\hrule
Your term paper for this course should be a scientific review that focuses on the coupling(s) between two or more components of the Earth system. You are free to choose the topic, but try to either focus on (1) the details of a specific coupling or feedback loop or (2) a specific, preferably regional scale, phenomenon (e.g., Asian monsoons). Due to the nature of the course, we are forced to cover a wide variety of topics in a short amount of time. The idea with this paper is for you to explore in detail some topic that you are interested in.

You will be required to submit a proposal for your term paper by the end of the first month of the semester. The purpose for this is to ensure that the topic you have selected is relevant to the course and is of appropriate scope. It will also help you get started on your paper early in the semester.

The paper should include, at a minimum, the following sections:

\begin{itemize}\itemsep -5pt
\item {\bf Title:} Make it specific, informative, and brief.
\item {\bf Abstract:} In 200 words or less (and yes, I will count the number of words), describe the key findings of the paper. Do not use a passive voice, and do not use the abstract to provide background material.
\item {\bf Introduction:} Provide background material on your topic and describe how it fits into Earth's climate system. State the goal or thesis of your paper. The introduction should be short (1--2 pages) and concise. In the introduction you are trying to bring readers up to speed with the ``thesis'' (i.e., objective) of the paper and explain why it is important. 
\item {\bf Body:} Synthesize ideas from the papers that you read to build a coherent thesis. Figure out your thesis before you start writing, and then build the discussion toward that thesis. This is where an outline will be very useful, as will judicious use of section headings.
\item {\bf Conclusions:} Restate the thesis of the paper and reiterate the major conclusions/ideas of the paper.
\end{itemize}

In terms of writing, I often find that its best to write the body of the paper first, followed by the introduction and conclusions. Save the title and abstract for the end once you know what exactly your paper is about. Do not take the title and abstract lightly. They are the most important part of any scientific article, because they are what readers will use to determine whether or not to continue reading the article. In terms of reading articles, most people will read the title and abstract, then look at figures, then read the conclusions and introduction. If they are really interested in the material they will also read the methods and results. Keep this in mind when writing your article. If you think something is important, put it somewhere that people will see it.

Use a consistent style for sections, captions, and references throughout the paper; I suggest mimicking the style of a scientific journal. The paper should be 15 pages double-spaced and in 12 pt font, not including references. Of the 15 pages, no more than 3 pages can be used for figures. The internet is a good source of information, especially when starting on a paper; however, your paper must rely exclusively on primary literature (articles from scientific journals and textbooks). You should reference \textit{at least} five scientific articles and preferably closer to ten articles. If you are unsure whether an article is primary literature, please ask.

Please carefully proofread your paper and include all cited references (but do not include articles in the references list that are not referenced in the paper). If you need help with writing, you can work with a tutor in the Writing Center. I am also happy to read through drafts of your paper and give you feedback, although I won't go through line-by-line to make sure that the grammar is perfect.

Being able to write clear, convincing arguments is extremely important in all fields of study. It is well worth your time to revise your term paper multiple times. Two final suggestions for making sure that your paper reads nicely: (1) read your paper aloud and (2) chop unnecessary words.

\clearpage
\hrulefill

Grading criteria for term paper proposals (10 pts):

\begin{itemize}\itemsep -5pt \squeezeup
  \item thesis/objective (3 pts)
  \item 1 page description of topic (4 pts)
  \item 3--5 references (3 pts)
\end{itemize}

\hrulefill

Grading rubric for term papers (100 pts total)

\begin{table}[h]
\begin{tabularx}{\textwidth}{p{1in}?X|X|X|}
  & \textbf{Exceeds Standards} & \textbf{Meets Standards} & \textbf{Standards Not Met} \\ \Xhline{2\arrayrulewidth}
  \textbf{Title and abstract} & Clear and succinct statement of findings (10 pts) & Ventures into background information or is not clear and to the point (7 pts) &  Contains too much introductory material and is not a statement of the findings (4 pts)\\ \hline
  \textbf{Introduction} & Clearly motivated by Earth system processes and couplings/feedbacks between processes \hspace{1cm}(10 pts) & Motivation is implied or not tightly connected to Earth system processes (7 pt) & Weak motivation, does not address system couplings and feedbacks \hspace{1cm}(4 pts) \\ \hline
  \textbf{Body} & Very clear, logical structure with strong arguments, factually correct (30 pts) & Some sections were disorganized, arguments don't build toward conclusions, some incorrect statements (20 pts) & Very disorganized and no clear arguments\hspace{1cm}(10 pts) \\ \hline
  \textbf{Conclusions} & Clearly summarized current state of knowledge, ties back to abstract and introduction, discusses future prospects (10 pts) & Contains key observations but doesn't synthesize results, discuss uncertainty, or future prospects (7 pts) & Poor summary that doesn't pull together the various threads of the paper (4 pts) \\ \hline
  \textbf{References} & Good use of primary literature, with extensive use of at least 5 peer-reviewed articles (10 pts) & Too few peer-reviewed articles or relied too heavily on non-peer-reviewed articles \hspace{1cm}(7 pts) & Less than five references used (4 pts)  \\ \hline
  \textbf{Overall writing} & Arguments are easy to follow, smooth transitions , grammatically correct, each paragraph has a distinct purpose (30 pts) & Some aspects are difficult to follow, many grammatical mistakes, or does not build toward a clear conclusion \hspace{1cm}(20 pts) & Numerous grammatical mistakes, poor organization, poorly structured (10 pts) \\ \hline
\end{tabularx}

\end{table}








\end{document}
